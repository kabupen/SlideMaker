
\usepackage{here, amsmath, latexsym, amssymb, bm, ascmac, mathtools, multicol, tcolorbox, subfig}
\usepackage{multirow}
\usepackage{graphicx}
\usepackage{comment}
\usepackage{amsmath}
\usepackage{atbegshi} % しおりの文字化け解消
\usepackage{xcolor}
\usepackage{xspace}
\usepackage{hyperref}

\usetheme{Luebeck}                % デザインの選択(省略可)
\usecolortheme{orchid}            % カラーテーマの選択(省略可)


% -------------------
% フォント
% -------------------
%usefonttheme{professionalfonts} % フォントテーマの選択(省略可)
%\usefonttheme{structuresmallcapsserif}
\usefonttheme{default}

\useinnertheme{circles}           % フレーム内のテーマの選択(省略可)
%\useoutertheme{infolines}        % フレーム外側のテーマの選択(省略可)
\setbeamertemplate{headline}{}    % ヘッダーを表示しない

% -------------------
% タイトルページ
% -------------------
\makeatletter
\defbeamertemplate*{title page}{supdefault}[1][]
{
  \vbox{}
  \vfill
  \begingroup
    \centering

    \begin{beamercolorbox}[sep=25pt,left,#1]{title}
      \usebeamerfont{title}
        \vskip0.25em%
      \vspace{15mm}
      \textbf{\inserttitle}\par%
      \vspace{2mm}
      \ifx\insertsubtitle\@empty\relax%
      \else%
        {\usebeamerfont{subtitle}\usebeamercolor[fg]{subtitle}\insertsubtitle\par}%
      \fi%
    \end{beamercolorbox}%

    \vskip1em\par

    \begin{beamercolorbox}[sep=8pt,left,#1]{author}
      \hspace{5mm}
      \usebeamerfont{author}\insertauthor
      \vspace{-5mm}
    \end{beamercolorbox}

    \begin{beamercolorbox}[sep=8pt,left,#1]{institute}
      \hspace{5mm}
      \vspace{-5mm}
      \usebeamerfont{institute}\insertinstitute
    \end{beamercolorbox}

    \begin{beamercolorbox}[sep=8pt,left,#1]{date}
      \hspace{5mm}
      \vspace{-5mm}
      \usebeamerfont{date}\insertdate
    \end{beamercolorbox}\vskip0.5em

    {\usebeamercolor[fg]{titlegraphic}\inserttitlegraphic\par}
  \endgroup
  \vfill
}
\setbeamertemplate{title page}[supdefault][colsep=-4bp,rounded=true,shadow=\beamer@themerounded@shadow]
\setbeamercolor{title}{fg=black}
\makeatother

% -------------------
% フッター
% -------------------
\setbeamertemplate{footline}[frame number] %スライド番号のみ表示
\setbeamercolor{section in head/foot}{fg=white, bg=darkgray}    %フッターの文字色、背景色
\setbeamerfont{footline}{size=\fontsize{5}{7}\selectfont}

\setbeamertemplate{footline}{
  % フッターの縁装飾
  \begin{beamercolorbox}[ht=5ex,leftskip=1.4cm,rightskip=.3cm]{author in head/foot} 
    \vspace{0.05cm}\hspace{-1cm}
    {\footnotesize \insertauthor} \hspace{3.2cm}  % 著者名
    {\footnotesize \insertdate}   \hspace{4cm}  % 日付
    {\footnotesize \insertframenumber}            % ページ番号
  \end{beamercolorbox}
}

% -------------------
% フレームの設定
% -------------------
\setbeamersize{text margin left=1.5em,text margin right=1em}
\setbeamercolor{frametitle}{fg=black, bg=} %フレームタイトルの色、背景色

\ifnum 42146=\euc"A4A2
\AtBeginShipoutFirst{\special{pdf:tounicode EUC-UCS2}}
\else
\AtBeginShipoutFirst{\special{pdf:tounicode 90ms-RKSJ-UCS2}}
\fi

\setbeamertemplate{navigation symbols}{} % ナビゲーションバー非表示
% \renewcommand{\kanjifamilydefault}{\gtdefault} %既定をゴシック体に
\renewcommand{\kanjifamilydefault}{mg} % 規定をヒラギノフォントに

% -------------------
% 各ページの設定
% -------------------
% タイトル
\setbeamercolor{title}{fg=structure, bg=} 

% 箇条書きスタイル、itemize
\setbeamertemplate{itemize item}{} % 見出しは何も付けない {\small\raise0.5pt\hbox{$\bullet$}}
\setbeamertemplate{itemize subitem}{\small\raise0.5pt\hbox{$\bullet$}}
\setbeamertemplate{itemize subsubitem}{\tiny\raise1.5pt\hbox{$\bullet$}}

\let\olditemize\itemize
\renewcommand{\itemize}{
    \olditemize
    \setlength{\itemsep}{2mm}
    \setlength{\parskip}{0mm}
    \setlength{\parsep}{5mm}
}

% color
\newcommand{\red}[1]{\textcolor{red}{#1}}
\newcommand{\green}[1]{\textcolor{green!40!black}{#1}}
\newcommand{\blue}[1]{\textcolor{blue!80!black}{#1}}


% -------------------
% マクロ定義
% -------------------
\newcommand{\subtitlepage}[1]{
  {
  \setbeamercolor{background canvas}{bg=darkgray}
  \setbeamercolor{frametitle}{fg=white} %フレームタイトルの色、背景色
  \begin{frame}[plain]{}
    \vspace{40mm}
    \centering
        % \textcolor{white}{\underline{\textbf{\huge{#1}}}}
        \textcolor{white}{\textbf{\huge{#1}}}
  \end{frame}
  }
}

\newcommand{\vspp}{\vspace{5mm}}
\newcommand{\vspm}{\vspace{-5mm}}
